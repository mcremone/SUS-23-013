The CMS apparatus~\cite{CMS:2008xjf,CMS:2023gfb} is a multipurpose, nearly hermetic detector, designed to trigger on~\cite{CMS:2020cmk,CMS:2016ngn,CMS:2024aqx} and identify electrons, muons, photons, and (charged and neutral) hadrons~\cite{CMS:2020uim,CMS:2018rym,CMS:2014pgm}. Its central feature is a superconducting solenoid of 6\unit{m} internal diameter, providing a magnetic field of 3.8\unit{T}. Within the solenoid volume are a silicon pixel and strip tracker, a lead tungstate crystal electromagnetic calorimeter (ECAL), and a brass and scintillator hadron calorimeter (HCAL), each composed of a barrel and two endcap sections. Forward calorimeters extend the pseudorapidity coverage provided by the barrel and endcap detectors. Muons are reconstructed using gas-ionization detectors embedded in the steel flux-return yoke outside the solenoid. More detailed descriptions of the CMS detector, together with a definition of the coordinate system used and the relevant kinematic variables, can be found in Refs.~\cite{CMS:2008xjf,CMS:2023gfb}.

The silicon tracker used in 2016 measured charged particles in the range $\abs{\eta} < 2.5$. For non-isolated particles of $1 < \pt < 10\GeV$ and $\abs{\eta} < 1.4$, the track resolutions were typically 1.5\% in \pt and 25--90 (45--150)\mum in the transverse (longitudinal) impact parameter~\cite{CMS:2014pgm}. At the beginning of 2017, a new pixel detector was installed~\cite{Phase1Pixel}; the upgraded tracker measured particles up to $\abs{\eta} = 3.0$ with typical resolutions of 1.5\% in \pt and 20--75\mum in the transverse impact parameter~\cite{DP-2020-049} for non-isolated particles of $1 < \pt < 10\GeV$. According to simulation studies~\cite{DP-2017-015}, similar improvements are expected in the longitudinal direction. The primary vertex (PV) is taken to be the vertex corresponding to the hardest scattering in the event, evaluated using tracking information alone, as described in Section 9.4.1 of Ref.~\cite{CMS-TDR-15-02}. 

In the region $\abs{\eta} < 1.74$, the HCAL cells have widths of 0.087 in pseudorapidity and 0.087 in azimuth ($\phi$). In the $\eta$-$\phi$ plane, and for $\abs{\eta} < 1.48$, HCAL cells map to $5{\times}5$ arrays of ECAL crystals to form calorimeter towers projecting radially outward from close to the nominal interaction point. For $\abs{\eta} > 1.74$, the coverage of the towers increases progressively to a maximum of 0.174 in $\Delta \eta$ and $\Delta \phi$. The forward hadron (HF) calorimeter uses steel as an absorber and quartz fibers as the sensitive material. The two halves of the HF are located 11.2\unit{m} from the interaction region, one at each end, and together they provide coverage in the range $3.0 < \abs{\eta} < 5.2$. They also serve as luminosity monitors.  

Events of interest are selected using a two-tier trigger system. The first level (L1), composed of custom hardware processors, uses information from the calorimeters and muon detectors to select events at a rate of around 100\unit{kHz} within a fixed latency of 4\mus~\cite{CMS:2020cmk}. The second level, known as the high-level trigger (HLT), consists of a farm of processors running a version of the full event reconstruction software optimized for fast processing, and reduces the event rate to a few kHz before data storage~\cite{CMS:2016ngn,CMS:2024aqx}.

A particle-flow (PF) algorithm~\cite{CMS:2017yfk} aims to reconstruct and identify each individual particle in an event, with an optimized combination of information from the various elements of the CMS detector. In this process, the identification of the PF candidate type (photon, electron, muon, and charged and neutral hadrons) plays an important role in the determination of the particle direction and energy. The energy of the photons is obtained from the ECAL measurement. The energy of electrons is determined from a combination of the electron momentum at the primary interaction vertex as determined by the tracker, the energy of the corresponding ECAL cluster, and the energy sum of all bremsstrahlung photons spatially compatible with originating from the electron track. The energy of muons is obtained from the curvature of the corresponding track. The energy of charged hadrons is determined from a combination of their momentum measured in the tracker and the matching ECAL and HCAL energy deposits, corrected for the response function of the calorimeters to hadronic showers. Finally, the energy of neutral hadrons is obtained from the corresponding corrected ECAL and HCAL energies. 

In this search, electrons (photons) are required to have $\pt>10$ (15)\GeV and $\abs{\eta}<2.5$. Muons are required to have $\pt>10\GeV$ and $\abs{\eta}<2.4$. All leptons and photons are required to be isolated. Isolation is calculated by imposing thresholds on the energy of PF candidates within a certain distance $\Delta R = \sqrt(\Delta\phi^2 + \Delta\eta^2)$ with respect to the lepton/photon. Additional selection criteria are applied to define ``loose'' (``veto'') electrons (muons and photons)~\cite{CMS:2020uim,CMS:2018rym}, which are used to reject unwanted events. Similarly, ``tight'' leptons/photons are defined and used to select events in control data samples. 

Hadronically decaying tau leptons are required to pass identification criteria using the hadron-plus-strips algorithm~\cite{Khachatryan:2015dfa}. In addition, a new algorithm for the identification of hadronic tau lepton decays, called \textsc{DeepTau}~\cite{deeptau}, is used. The \textsc{DeepTau} algorithm is based on a multi-classification technique that serves to discriminate genuine hadronic tau lepton decays from jets, electrons, and muons. In addition, the overlap between tau leptons and electrons or muons is further suppressed by removing tau leptons that lie within a distance $\Delta R\ < $ 0.4 of a well-reconstructed and isolated electron or muon.

For each event, hadronic jets are clustered from the PF candidates using the infrared and collinear safe anti-\kt algorithm~\cite{Cacciari:2008gp, Cacciari:2011ma} with a distance parameter of 0.4 or 1.5. 
Depending on the respective distance parameter, these jets are referred to as ``AK4'' or ``AK15'' jets.
Jet momentum is determined as the vectorial sum of all particle momenta in the jet, and is found from simulation to be, on average, within 5 to 10\% of the true momentum over the whole \pt spectrum and detector acceptance. Jet energy corrections are derived from simulation to bring the measured response of jets to that of particle-level jets on average. In situ measurements of momentum balance in dijet, $\text{photon} + \text{jet}$, $\PZ + \text{jet}$, and multijet events are used to account for any residual differences in the jet energy scale between data and simulation~\cite{CMS:2016lmd}. The jet energy resolution typically amounts to 15--20\% at 30\GeV, 10\% at 100\GeV, and 5\% at 1\TeV~\cite{CMS:2016lmd}. Additional selection criteria are applied to each jet to remove jets potentially dominated by anomalous contributions from various subdetector components or reconstruction failures.

Additional proton-proton interactions within the same or nearby bunch crossings (pileup) can contribute additional tracks and calorimetric energy depositions to the jet momentum. %%% AK4 jets are chs, are recommended for pre-legacy Run2
To mitigate this effect in the AK4 jets, charged particles identified to be originating from pileup vertices are discarded, and an offset correction is applied to correct for remaining contributions. For AK15 jets, the pileup per particle identification algorithm (PUPPI) ~\cite{Sirunyan:2020foa,Bertolini:2014bba} is used to mitigate the effect of pileup at the reconstructed particle level, making use of local shape information, event pileup properties, and tracking information. A local shape variable is defined, which distinguishes between collinear and soft diffuse distributions of other particles surrounding the particle under consideration. The former is attributed to particles originating from the hard scatter, and the latter is attributed to particles originating from pileup interactions. Charged particles identified to be originating from pileup vertices are discarded. For each neutral particle, a local shape variable is computed using the surrounding charged particles compatible with the primary vertex within the tracker acceptance ($\abs{\eta} < 2.5$), and using both charged and neutral particles in the region outside of the tracker coverage. The momenta of the neutral particles are then rescaled according to their probability to originate from the primary interaction vertex deduced from the local shape variable, superseding the need for jet-based pileup corrections~\cite{Sirunyan:2020foa}. 


The AK4 jets used in this search are further required to have a \pt larger than 30 GeV and $|\eta| < 2.5$. Jets with $\Delta R\ <$ 0.4 with respect to a well identified and isolated lepton or photon are removed. To identify jets originated by the hadronization of b quarks (hereafter referred to as ``b jets''), the \textsc{DeepJet} algorithm~\cite{Bols:2020bkb} is employed. A loose working point is used, defined for each year of data-taking as the minimum requirement in the \textsc{DeepJet} discriminator distribution, which returns a 10\% rate of misidentifying a jet originated by a light-flavor quark. The loose working point corresponds to an efficiency of correctly identifying jets originated by b quarks (i.e., b-tagging efficiency) of 90--95\%, depending on the \pt of the AK4 jet.

The AK15 jets used in this search are further required to have a \pt larger than 160 GeV and $|\eta| < 2.4$. Jets with $\Delta R <$ 1.5 with respect to a well-identified and isolated lepton or photon are removed. The modified mass drop tagger algorithm~\cite{Dasgupta:2013ihk,Butterworth:2008iy}, also known as the soft-drop (SD) algorithm, with the angular exponent $\beta = 0$, soft cutoff threshold $z_{\text{cut}} < 0.1$, and characteristic radius $R_{0} = 1.5$~\cite{Larkoski:2014wba}, is applied to remove soft, wide-angle radiation from the jet. 
To identify AK15 jets that are consistent with the hadronization of a $b\bar{b}$ pair from the decay of a boosted massive resonance, the  \textsc{DeepAK15} algorithm~\cite{Sirunyan:2020lcu} is used. More specifically, the ``mass-decorrelated'' (MD) version of \textsc{DeepAK15} is employed. In this variant, an adversarial training is performed in which a second neural network is made to extract the AK15 jet mass from the output of the \textsc{DeepAK15} graph neural network. A good performance of this second network yields to a penalty on the joint cost function of the two networks. Therefore, this method optimizes the ability to correctly identify the origin of an AK15 jet while systematically decorrelating the output score from the AK15 jet mass. This approach avoids shaping the AK15 jet mass distribution in background events. Since the strategy for this search relies on the AK15 jet mass shape for background estimation, the MD version of the tagger offers the best option. Identification working points are defined for each year of data-taking and optimized for this specific search. 

The missing transverse momentum vector \ptvecmiss is computed as the negative vector sum of the transverse momenta of all the PF candidates in an event, and its magnitude is denoted as \ptmiss. The \ptvecmiss is modified to account for corrections to the energy scale and resolution of the reconstructed AK4 jets in the event~\cite{Sirunyan:2019kia}.  Anomalous high-\ptmiss events can be due to a variety of reconstruction failures, detector malfunctions, or non-collision backgrounds. Such events are rejected by dedicated filters that are designed to eliminate more than 85-- 90\% of spurious high-\ptmiss events with a signal efficiency exceeding 99.9\%~\cite{Sirunyan:2019kia}. 

Together with the AK15 jet mass, the hadronic recoil, called U, is also used to distinguish the signal from the backgrounds. It represents the total transverse momentum of all the non-hadronic particles in each event. The signal events in this search contain only jets and no other reconstructed candidates, therefore U is equivalent to \ptmiss of the event. For the leading background processes, identified in Section~\ref{selection}, U corresponds to the \pt of a vector boson. In those data samples of this search where no lepton is required, the lepton from the vector boson decay is missed and U is once again equivalent to the \ptmiss of the event. In data samples where a lepton is required in the final state, the lepton from the vector-boson decay is identified and U is equivalent to the magnitude of the vector sum of \ptvecmiss and the lepton \ptvec.

%%%transverse momentum of the hadronic recoil, called U. For the leading background processes, this also corresponds to the transverse momentum of a vector boson. In the control data samples of this search where a lepton in the final state is required, \ptmiss is not equivalent to U anymore. At the same time, U still represents the transverse momentum of the vector boson from those background process that populate these samples. Since this search relies on prediction derived from data in control samples to constrain the main backgrounds, the U derived in the control samples can be used to model the transverse momentum of the vector boson. In the control data samples, the variable is computed subtracting from \ptvecmiss the $\ptvec$ of the lepton. %%%
